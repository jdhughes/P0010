% !TEX TS-program = pdflatex
% !TEX encoding = UTF-8 Unicode

% This is a simple template for a LaTeX document using the "article" class.
% See "book", "report", "letter" for other types of document.

\documentclass[11pt]{article} % use larger type; default would be 10pt

\usepackage[utf8]{inputenc} % set input encoding (not needed with XeLaTeX)

%%% Examples of Article customizations
% These packages are optional, depending whether you want the features they provide.
% See the LaTeX Companion or other references for full information.

%%% PAGE DIMENSIONS
\usepackage{geometry} % to change the page dimensions
\geometry{letterpaper} % or letterpaper (US) or a5paper or....
\geometry{margin=1in} % for example, change the margins to 2 inches all round
% \geometry{landscape} % set up the page for landscape
%   read geometry.pdf for detailed page layout information

\usepackage{graphicx} % support the \includegraphics command and options

% \usepackage[parfill]{parskip} % Activate to begin paragraphs with an empty line rather than an indent

%%% PACKAGES
\usepackage{amsmath}
\usepackage{booktabs} % for much better looking tables
\usepackage{array} % for better arrays (eg matrices) in maths
\usepackage{paralist} % very flexible & customisable lists (eg. enumerate/itemize, etc.)
\usepackage{verbatim} % adds environment for commenting out blocks of text & for better verbatim
\usepackage{subfig} % make it possible to include more than one captioned figure/table in a single float
% These packages are all incorporated in the memoir class to one degree or another...

%%% HEADERS & FOOTERS
\usepackage{fancyhdr} % This should be set AFTER setting up the page geometry
\pagestyle{fancy} % options: empty , plain , fancy
\renewcommand{\headrulewidth}{0pt} % customise the layout...
\lhead{}\chead{}\rhead{}
\lfoot{}\cfoot{\thepage}\rfoot{}

%%% SECTION TITLE APPEARANCE
\usepackage{sectsty}
\allsectionsfont{\sffamily\mdseries\upshape} % (See the fntguide.pdf for font help)
% (This matches ConTeXt defaults)

%%% ToC (table of contents) APPEARANCE
\usepackage[nottoc,notlof,notlot]{tocbibind} % Put the bibliography in the ToC
\usepackage[titles,subfigure]{tocloft} % Alter the style of the Table of Contents
\renewcommand{\cftsecfont}{\rmfamily\mdseries\upshape}
\renewcommand{\cftsecpagefont}{\rmfamily\mdseries\upshape} % No bold!

%%% END Article customizations

%%% The "real" document content comes below...

\title{Brief Article}
\author{The Author}
%\date{} % Activate to display a given date or no date (if empty),
         % otherwise the current date is printed 

\begin{document}
\maketitle

\section{First section}

For a one-dimension model where NROW = NLAY = 1:

\begin{align}  
C_{i-\frac{1}{2},j,k} \left( h^t_{i-1,j,k} - h^t_{i,j,k} \right) &+ C_{i+\frac{1}{2}} \left( h^t_{i+1,j,k} - h^t_{i,j,k} \right) + H_{i}\left( h^{t}_{b} - h^{t}_{i,j,k} \right) \nonumber \label{eq1d1}\\ &+ \frac{S^{t}_{i,j,k} \Delta x_{i,j,k} \Delta y_{i,j,k} \Delta z_{i,j,k}}{\Delta t} \left( h^{t+1}_{i,j,k} - h^t_{i,j,k} \right) + w = 0
\end{align}

\noindent simplify equation \ref{eq1d1} to:

\begin{align}
h^t_{i,j,k} - h^{t+1}_{i,j,k} = \frac{\Delta t}{S^{t}_{i,j,k} \Delta x_{i,j,k} \Delta y_{i,j,k} \Delta z_{i,j,k}} &\left[ C_{i-\frac{1}{2},j,k} \left( h^t_{i-1,j,k} - h^t_i \right) + C_{i+\frac{1}{2},j,k} \left( h^t_{i+1,j,k} - h^t_{i,j,k} \right) \right.\nonumber  \label{eq1d2} \\ &\qquad \left.+ H_{i}\left( h^{t}_{b} - h^{t}_{i,j,k} \right) + w_{i,j,k} \right] 
\end{align}

\noindent Set:

\begin{equation} \label{eqnTS}
\hat{s}^{t}_{i,j,k} = \frac{\Delta t}{S_{i,j,k} \Delta x_{i,j,k} \Delta y_{i,j,k} \Delta z_{i,j,k}}
\end{equation}


\noindent rearrange equation \ref{eq1d2} to:

\begin{align}
h^{t+1}_{i,j,k} = h^t_{i,j,k} - \hat{s}^{t}_{i,j,k} &\left[ C_{i-\frac{1}{2},j,k} \left( h^t_{i-1,j,k} - h^t_{i,j,k} \right) \right.\nonumber  \label{eq1d3} \\ &\qquad \left.+ C_{i+\frac{1}{2},j,k} \left( h^t_{i+1,j,k} - h^t_{i,j,k} \right) - H_{i} h^{t}_{i,j,k} +  H_{i,j,k} h^{t}_{b} + w_{i,j,k} \right] 
\end{align}

Adding known boundary terms:

\begin{equation} \label{eq1d4}
r^{t}_{i} = H_{i,j,k} h^{t}_{b} + w_{i,j,k} 
\end{equation}


Equation \ref{eq1d3} in matrix-vector form:

\begin{equation} \label{eq1dvec}
\mathbf{h}^{t+1} = \mathbf{h}^t - \hat{\mathbf{s}} \left[ \left ( \mathbf{C} - \mathbf{H} \right ) \mathbf{h}^{t} + \mathbf{r} \right] 
\end{equation}

\noindent Subject to the stability constraint:

\begin{equation} \label{eq1dstab1}
max \left( \frac{C_{ijk} \Delta t}{S_{ijk} \Delta x \Delta y \Delta z} \right) \le 1 
\end{equation}

\noindent rearrange equation \ref{eq1dstab1} to determine the maximum stable time step length:

\begin{equation} \label{eq1dstab2}
\Delta t_{max} = \frac{S_{ijk} \Delta x \Delta y \Delta z}{C_{ijk} } 
\end{equation}

\subsection{A subsection}

More text.

\end{document}
